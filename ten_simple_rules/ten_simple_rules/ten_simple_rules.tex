% Template for PLoS
% Version 3.4 January 2017
\documentclass[10pt,letterpaper]{article}
\usepackage[top=0.85in,left=2.75in,footskip=0.75in]{geometry}

% amsmath and amssymb packages, useful for mathematical formulas and symbols
\usepackage{amsmath,amssymb}

% Use adjustwidth environment to exceed column width (see example table in text)
\usepackage{changepage}

% Use Unicode characters when possible
\usepackage[utf8x]{inputenc}

% textcomp package and marvosym package for additional characters
\usepackage{textcomp,marvosym}

% cite package, to clean up citations in the main text. Do not remove.
% \usepackage{cite}

% Use nameref to cite supporting information files (see Supporting Information section for more info)
\usepackage{nameref,hyperref}

% line numbers
\usepackage[right]{lineno}

% ligatures disabled
\usepackage{microtype}
\DisableLigatures[f]{encoding = *, family = * }

% color can be used to apply background shading to table cells only
\usepackage[table]{xcolor}

% array package and thick rules for tables
\usepackage{array}

% create "+" rule type for thick vertical lines
\newcolumntype{+}{!{\vrule width 2pt}}

% create \thickcline for thick horizontal lines of variable length
\newlength\savedwidth
\newcommand\thickcline[1]{%
  \noalign{\global\savedwidth\arrayrulewidth\global\arrayrulewidth 2pt}%
  \cline{#1}%
  \noalign{\vskip\arrayrulewidth}%
  \noalign{\global\arrayrulewidth\savedwidth}%
}

% \thickhline command for thick horizontal lines that span the table
\newcommand\thickhline{\noalign{\global\savedwidth\arrayrulewidth\global\arrayrulewidth 2pt}%
\hline
\noalign{\global\arrayrulewidth\savedwidth}}


% Remove comment for double spacing
%\usepackage{setspace}
%\doublespacing

% Text layout
\raggedright
\setlength{\parindent}{0.5cm}
\textwidth 5.25in
\textheight 8.75in

% Bold the 'Figure #' in the caption and separate it from the title/caption with a period
% Captions will be left justified
\usepackage[aboveskip=1pt,labelfont=bf,labelsep=period,justification=raggedright,singlelinecheck=off]{caption}
\renewcommand{\figurename}{Fig}

% Use the PLoS provided BiBTeX style
% \bibliographystyle{plos2015}

% Remove brackets from numbering in List of References
\makeatletter
\renewcommand{\@biblabel}[1]{\quad#1.}
\makeatother

% Leave date blank
\date{}

% Header and Footer with logo
\usepackage{lastpage,fancyhdr,graphicx}
\usepackage{epstopdf}
\pagestyle{myheadings}
\pagestyle{fancy}
\fancyhf{}
\setlength{\headheight}{27.023pt}
\lhead{\includegraphics[width=2.0in]{PLOS-submission.eps}}
\rfoot{\thepage/\pageref{LastPage}}
\renewcommand{\footrule}{\hrule height 2pt \vspace{2mm}}
\fancyheadoffset[L]{2.25in}
\fancyfootoffset[L]{2.25in}
\lfoot{\sf PLOS}

%% Include all macros below
\newcommand{\lorem}{{\bf LOREM}}
\newcommand{\ipsum}{{\bf IPSUM}}





\usepackage{forarray}
\usepackage{xstring}
\newcommand{\getIndex}[2]{
  \ForEach{,}{\IfEq{#1}{\thislevelitem}{\number\thislevelcount\ExitForEach}{}}{#2}
}

\setcounter{secnumdepth}{0}

\newcommand{\getAff}[1]{
  \getIndex{#1}{York University}
}

\providecommand{\tightlist}{%
  \setlength{\itemsep}{0pt}\setlength{\parskip}{0pt}}

\begin{document}
\vspace*{0.2in}

% Title must be 250 characters or less.
\begin{flushleft}
{\Large
\textbf\newline{Ten simple principles for reverse-engineering reproducible solutions to
environmental management challenge cases.} % Please use "sentence case" for title and headings (capitalize only the first word in a title (or heading), the first word in a subtitle (or subheading), and any proper nouns).
}
\newline
\\
Christopher J. Lortie\textsuperscript{\getAff{Biology, York University}}\textsuperscript{*},
Malory Owen\textsuperscript{\getAff{York University}}\\
\bigskip
\textbf{\getAff{York University}}Biology, 4700 Keele St.~Toronto, ON, Canada, M3J1P3\\
\bigskip
* Corresponding author: lortie@yorku.ca\\
\end{flushleft}
% Please keep the abstract below 300 words
\section*{Abstract}
An environmental management challenge case is an opportunity to use
fundamental science to inform evidence-based decisions for the
environment. Contemporary science is embracing open science and
increasingly conscious of reproduciblility. Synergistically, applying
these two paradigms in concert advances our capacity to move beyond
context dependency and singlular thinking to reverse engineer solutions
from published scientific evidence associated with one challenge to
many. Herein, we provide a short list of principles that can guide those
that seek solutions and syntheses to address environmental management
through primary scientific literature.

% Please keep the Author Summary between 150 and 200 words
% Use first person. PLOS ONE authors please skip this step.
% Author Summary not valid for PLOS ONE submissions.
\section*{Author summary}
Grand challenges require grand solutions. Environmental management
cannot neglect fundamental science as a substrate for effective decision
making.

\linenumbers

% Use "Eq" instead of "Equation" for equation citations.
\section{Introduction}\label{introduction}

Context. Definitions. Link between fundamental scientific inquiry and
environmental management. Ten simple priniciples are proposed as a mean
to promote engagement with scientific literature to generalize
solutions.

\section{The principles}\label{the-principles}

\begin{enumerate}
\def\labelenumi{\arabic{enumi}.}
\item
  Reframe the problem as challenge.
\item
  Describe the scope and extent of the challenge.
\item
  Explicitly link the basic science to management implications and
  policy.
\item
  Propose implications of ignoring this challenge.
\item
  State the direct human needs associated with this challenge.
\item
  List at least one limitation of the study and explain.
\item
  Explore the benefits of minimal intervention for environmental
  managers and stakeholders.
\item
  List the tools applied to this challenge.
\item
  Explain the role that the primary tool addressed for the challenge -
  i.e.~identification/research evidence, management/solution applied, or
  inform policy.
\item
  Apply the tool to another challenge or explain how it is general and
  scaleable.
\end{enumerate}

\section{Implications}\label{implications}

Big picture thinking. Creative reuse of solutions from one domain to
another. Functional use of scientific literature. Goals-directed
engagement with scientific evidence.

\section*{References}\label{references}
\addcontentsline{toc}{section}{References}

\nolinenumbers


\end{document}

